\chapter{Theory}
\label{cha:theory}

\section{Visual Search}



\section{Reinforcement Learning}

Reinforcement learning (RL) is a subfield of machine learning concerned intelligent agents that learn to achieve some goal through interaction with their environment. An agent conditioned to improve its behaviour through reward and punishment. In this section, some key concepts will be introduced.

\subsection{Markov Decision Process}

The problem of learning from interaction to achieve a goal is usually framed as a (finite) Markov Decision Process (MDP). An \textit{agent} learns by interacting with an \textit{environment}. At each discrete time step, the agent selects an action and perceives an observation of the state of its environment as well as a reward signal. An MDP is formally defined as a quadruple \(\langle S, A, R, P, \rho_0 \rangle\), where

\begin{itemize}
    \item \(S\) is the state space,
    \item \(A\) is the action space,
    \item \(X\)
    \item \(R : S \times A \times S \rightarrow \mathbb{R}\) is the reward function, 
\end{itemize}

\subsection{Taxonomy of Algorithms}

\begin{itemize}
    \item Model-free vs. model-based
    \item 
\end{itemize}


\subsection{Policy Optimization}



\subsection{Generalization}