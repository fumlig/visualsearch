\chapter{Conclusion}
\label{cha:conclusion}

% This chapter contains a summarization of the purpose and the research
% questions. To what extent has the aim been achieved, and what are the
% answers to the research questions?
% 
% The consequences for the target audience (and possibly for researchers
% and practitioners) must also be described. There should be a section
% on future work where ideas for continued work are described. If the
% conclusion chapter contains such a section, the ideas described
% therein must be concrete and well thought through.

Visual search is ubiquitous in our daily lives as humans.
Automated visual search systems therefore naturally have many potential applications.

In this work, we have presented a method for jointly learning control of visual attention, recognition and localization using deep reinforcement learning.

Bajcsy, Aloimonos and Tsotsos~\cite{bajcsy_revisiting_2018} connect past work in active vision with recent advances in robotics, artificial intelligence and computer vision.
They argue that a complete artificial agent must include active perception.
The goal of artificial intelligence research is the computational generation of intelligent behavior.
Agents that choose their behavior based on their context and know why they behave as they do would certainly seem to embody this.
In this work, we have introduced such an agent.

\section{Future Work}

\subsection{Difficult Detection Problems}

\dots

\subsection{Ablation Studies}

\dots

\subsection{Visual Search with Reinforcement Learning}

We suggest that future work utilizing reinforcement learning for visual search should\dots
