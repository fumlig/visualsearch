\chapter{Conclusion}
\label{cha:conclusion}

Visual search is ubiquitous in our daily lives as humans.
Automated visual search systems therefore naturally have many potential applications.
\dots

In this work, we have presented a method for jointly learning control of visual attention, recognition and localization using deep reinforcement learning.
\dots

\begin{itemize}
    \item Relate back to research questions.
    \item Answer them explicitly.
\end{itemize}

%Bajcsy, Aloimonos and Tsotsos~\cite{bajcsy_revisiting_2018} connect past work in active vision with recent advances in robotics, artificial intelligence and computer vision.
%They argue that a complete artificial agent must include active perception.
%The goal of artificial intelligence research is the computational generation of intelligent behavior.
%Agents that choose their behavior based on their context and know why they behave as they do would certainly seem to embody this.
%In this work, we have introduced such an agent.

\section{Future Work}

\subsection{Realistic Search Tasks}

\begin{itemize}
    \item Real search problems are likely to be more difficult to learn.
    \item There is a lack of datasets and environments for this (AI2 THOR, object detection in large images)
\end{itemize}

\subsection{Ablation Studies}

\begin{itemize}
    \item Evaluate the importance of image observations, position observations and memory.
    \item We could draw some conclusions regarding these from our results\dots
\end{itemize}