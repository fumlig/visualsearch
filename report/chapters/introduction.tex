\chapter{Introduction}
\label{cha:introduction}

In this thesis project, the problem of searching for targets in unknown environments is addressed. This chapter presents the motivation behind the project, the research questions that are addressed, and the delimitations. 

\section{Motivation}
\label{sec:motivation}

Searching for targets (objects of interest, goals) in unknown environments is a well-studied problem that appears in many areas from robotics~\cite{}, computer vision~\cite{}. Applications 

In this project, an instance of this problem is considered where a pan-tilt-zoom camera is fixed in place, and used to scan an environment for target objects. Automating this task is of interest for multiple reasons


While 

This is where the studied problem is described from a general
point of view and put in a context which makes it clear that
it is interesting and well worth studying. The aim is to make
the reader interested in the work and create an urge to
continue reading.

\section{Aim}
\label{sec:aim}

The aim of this thesis is to implement an autonomous agent that intelligently searches its environment for targets. Ideally, the agent should learn common characteristics of environments and utilize this knowledge to search for targets in new environments more effectively.

\section{Research questions}
\label{sec:research-questions}

This thesis will address the following questions:

\begin{enumerate}
  \item How can the visual search problem be solved by a learning agent?
  \item How can a simulator that tests the ability of an agent to solve  the presented problem be implemented?
  \item How does the learning agent compare to common non-learning methods?
\end{enumerate}

\section{Delimitations}
\label{sec:delimitations}

This thesis will be focused on the behavioural aspects of the presented problem. To train and test agents, a simplified environment will be used. This will test the desired characteristics of the agent as presented above, but will not simulate realistic environments.