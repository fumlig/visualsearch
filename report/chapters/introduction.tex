\chapter{Introduction}
\label{cha:introduction}

% unknown but familiar

In this thesis project, the problem of searching for targets in unknown but familiar environments is addressed. This chapter presents the motivation behind the project, the research questions that are addressed, and the delimitations. 

\section{Motivation}
\label{sec:motivation}

Searching for targets (objects of interest, goals) in unknown environments is a well-studied problem that appears in many areas ranging from robotics~\cite{} to computer vision~\cite{} and neuroscience~\cite{}. Applications include search and rescue, 

In many cases, the environment is familiar in that it has characteristics that are similar to previously seen environments.

% familiarity effect? Target familiarity and visual working memory do not influence familiarity effect in visual search

This is similar to the exploration problem, where a robot is tasked with maximizing the knowledge of a certain area.

% difference to eyes: camera can not move instantly (we only have head movements)

% also similar to search and rescue

If the environment is known, this problem becomes drastically easier.

In this project, an instance of the target search problem is considered where the environment is searched by a pan-tilt camera fixed in place. The camera has a limited view of the environent. Automating this task is of interest for multiple reasons. Manually controlling a camera may be costly, and the performance of a human operator may be suboptimal. Crucial to the problem is generalization.

\section{Aim}
\label{sec:aim}

The aim of this thesis is to implement and evaluate an autonomous agent that intelligently searches its environment for targets. The agent should learn common characteristics of environments and utilize this knowledge to search for targets in new environments more effectively. 

\section{Research questions}
\label{sec:research-questions}

This thesis will address the following questions:

\begin{enumerate}
  \item How can the visual search problem be solved by a learning agent?
  \item How can a simulator that tests the ability of an agent to solve the presented problem be implemented?
  \item How does the learning agent compare to common non-learning methods?
\end{enumerate}

\section{Delimitations}
\label{sec:delimitations}

This thesis will be focused on the behavioural aspects of the presented problem. To train and test agents, a simplified environment will be used. This will test the desired characteristics of the agent as presented above, but will not simulate realistic environments.