% should decide on a clear terminology,
% target, distractor, region, view, move, etc...

\chapter{Introduction}
\label{cha:introduction}

% ~3p

In this thesis project, the problem of searching for targets in unknown but familiar environments is addressed.
This chapter presents the motivation behind the project, the research questions that are addressed, and the delimitations. 

\section{Motivation}
\label{sec:motivation}

% This is where the studied problem is described from a general
% point of view and put in a context which makes it clear that
% it is interesting and well worth studying. The aim is to make
% the reader interested in the work and create an urge to
% continue reading.

The ability to visually search for things in an environment is fundamental to intelligent behaviour.
We humans are constantly looking for things, be it be it the right book in the bookshelf, a certain keyword in an article or blueberries in the forest.
In many cases, it is important that this search is strategic, efficient, and fast.
Animals need to quickly identify predators, and drivers need to be able to search for pedestrians crossing the road they are driving on.

An intelligent searcher should be able to 

Automating the task of searching is of great interest

While searching for targets is often seemingly effortless to humans, it is a complex process.
How humans and animals search for things has been extensively studied in neuroscience and neurobiology~\cite{eckstein_visual_2011,wolfe_visual_2010,wolfe_guided_2021}.

Applications such as helping robots and search and rescue mean that it is of great interest to automate visual search.
In the computer vision field, there has been several attempts to mimic the way humans search in machines~\cite{}.
Most attempts focus on fully observable scenes where the target is in view and the task is to localize it (object localization).
However, in many real-world visual search scenarios the field-of-view is limited.
This means that the search process is split into two steps: directing the field of view (covert attention), and locating targets within the view (overt attention).
Much work has been focused on latter, locating targets within the field of view~\cite{}.

When only a fraction of the environment is visible, where to move the field of view becomes an important decision.
The characteristics of the searched environment can often be used to find targets quicker.
For example, if one is foraging for blueberries it makes sense to search the ground rather than the trees.
Similarly, if one is searching a satellite image for boats it is reasonable to focus on ocean shores.
If you see a railroad track or the wake of a boat you can usually follow it to find a vehicle.
The exact characteristics of the environment need not be constant - forests with blueberries can vary greatly in appearance and boats can be found in all of the seven seas.
In many cases, the environment is familiar in that it has characteristics that are similar to previously seen environments.
Humans are able to generalize in such cases.

Manually creating search algorithms for such tasks is problematic.
The appearance and distribution of targets in an environment varies greatly, and may be subtle.
The visual richness of the environment itself is another problem.
How can you identify useful hints from the environment to guide covert attention?
Doing so manually can be labour intensive, especially if a searcing system should be deployed in many different environments.
If one could instead learn the underlying from a limited set of sample environments and generalize to unseen similar environments this problem would be circumvented.

% Generalization
% Pan-tilt-zoom

Deep reinforcement learning is an approach for how to act.
It has been applied to a number of problems with sucess\dots

\section{Aim}
\label{sec:aim}

% Pan-Tilt-Zoom setup

% What is the underlying purpose of the thesis project?

The aim of this thesis is to investigate how an intelligent agent that learns to search for targets can be implemented with deep reinforcement learning.
Such an agent should learn the characteristics of the environments it is trained on and utilize this knowledge to effectively search for targets in unseen environments.
Specifically, we consider scenarious where the agent can only observe a small portion of its environment at any given time.
The agent has to actively choose where to look in order to gain new information about the environment.

We are looking for the following properties:

\begin{itemize}
  \item The agent should prioritize regions where the probability of finding a target is high according to previous experience.
  \item The agent should utilize information from previously visited regions to decide where to move.
  \item The agent should be able to find multiple targets while minimizing its path length.
  \item The agent should avoid looking at the same region twice.
\end{itemize}

Our contributions are as follows:

\begin{itemize}
  \item \dots
\end{itemize}

\section{Research questions}
\label{sec:research-questions}

% This is where the research questions are described.
% Formulate these as explicit questions, terminated with a
% question mark. A report will usually contain several different
% research questions that are somehow thematically connected.
% There are usually 2-4 questions in total.
% 
% Examples of common types of research questions (simplified
% and generalized):
% 
% \begin{enumerate}
% \item How does technique X affect the possibility of achieving the
%   effect Y?
% 
% \item How can a system (or a solution) for X be realized so
%   that the effect Y is achieved?
% 
% \item What are the alternatives to
%   achieving X, and which alternative gives the best effect considering
%   Y and Z? (This research question is normally broken down in to 2
%   separate questions.)
% 
% \end{enumerate}
% 
% 
% Observe that a very specific research question almost always
% leads to a better thesis report than a general research question
% (it is simply much more difficult to make something good
% from a general research question.)
% 
% The best way to achieve a really good and specific research
% question is to conduct a thorough literature review and get
% familiarized with related research and practice. This leads to
% ideas and terminology which allows one to express oneself
% with precision and also have something valuable to say in the
% discussion chapter. And once a detailed research question
% has been specified, it is much easier to establish a suitable
% method and thus carry out the actual thesis work much faster
% than when starting with a fairly general research question. In
% the end, it usually pays off to spend some extra time in the
% beginning working on the literature review. The thesis
% supervisor can be of assistance in deciding when the research
% question is sufficiently specific and well-grounded in related
% research.

This thesis will address the following questions:

\begin{enumerate}
  \item \label{itm:rq1} How can a learning agent that learns to intelligently search for targets be implemented?
  \item \label{itm:rq2} How does the learning agent compare to random walk, exhaustive search, a human searcher?
  \item \label{itm:rq3} How well does the learning agent generalize to unseen but familiar environments?
\end{enumerate}

\section{Delimitations}
\label{sec:delimitations}

% This is where the main delimitations are described. For
% example, this could be that one has focused the study on a
% specific application domain or target user group. In the
% normal case, the delimitations need not be justified.

This thesis will be focused on the behavioral aspects of the presented problem.
We do not focus on difficult detection problems, but rather efficient actions.
For this reason, targets will deliberately be made easy to detect.
For simplicity, we make the assumption that the searched environment is static.
The appearance of the environment and the location of the targets does not change from one observation to the next.
