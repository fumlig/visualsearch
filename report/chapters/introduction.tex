\chapter{Introduction}
\label{cha:introduction}

In this thesis project, the problem of searching for targets in unknown but familiar environments is addressed.
This chapter presents the motivation behind the project, the research questions that are addressed, and the delimitations. 

% there is extensive research in visual search.
% can a learning agent exhibit these behaviours?
% can also downplay visual aspect
% specify actions properly: how do they transform view, and what does the trigger correspond to? 

\section{Motivation}
\label{sec:motivation}

% This is where the studied problem is described from a general
% point of view and put in a context which makes it clear that
% it is interesting and well worth studying. The aim is to make
% the reader interested in the work and create an urge to
% continue reading.

The ability to visually search for targets in an environment is crucial to many parts of our daily lives.
We are constantly looking for things, be it the right book in the bookshelf, a certain keyword in an article or blueberries in the forest.
In many cases, it is important that this search is efficient and fast.
Animals need to quickly identify predators, and drivers need to be able to search for pedestrians crossing the road they are driving on.

While searching for targets is often seemingly effortless to humans, it is a complex process.
How humans and animals search for things has been extensively studied in neuroscience and neurobiology~\cite{eckstein_visual_2011,wolfe_visual_2010,wolfe_guided_2021}.
Applications from automated search and rescue to helping robots mean that it is of great interest to automate visual search.
In the computer vision field, there has been several attempts to mimic the way humans search in machines~\cite{}.
Most attempts focus on fully observable scenes where the target is in view and the task is to localize it (object localization).
However, in many real-world visual search scenarios the field-of-view is limited.
This means that the search process is split into two steps: directing the field of view (covert attention), and locating targets within the view (overt attention).
Much work has been focused on latter, locating targets within the field of view~\cite{}. 

When only a fraction of the environment is visible, where to move the field of view becomes an important decision.
The characteristics of the searched environment can often be used to find targets quicker.
For example, if one is foraging for blueberries it makes sense to search the ground rather than the trees.
Similarly, if one is searching a satellite image for boats it is reasonable to focus on ocean shores.
If you see a railroad track or the wake of a boat you can usually follow it to find a vehicle.
The exact characteristics of the environment need not be constant - forests with blueberries can vary greatly in appearance and boats can be found in all of the seven seas.
In many cases, the environment is familiar in that it has characteristics that are similar to previously seen environments.
Humans are able to generalize in such cases.

Manually creating search algorithms for such tasks is problematic.
The appearance and distribution of targets in an environment varies greatly, and may be subtle.
The visual richness of the environment itself is another problem.
How can you identify useful hints from the environment to guide covert attention?
Manually engineering such a platform seems infeasible.
If one could instead learn the underlying from a limited set of sample environments and generalize to unseen similar environments this problem would be circumvented.

% we want a system that can
% - work in familiar environments
% - learns effective scanning patterns
% - in the best case better than exhaustive
% - integrates history (features, location, etc)

% - train: show the agent some samples of previous objects and their locations
% - test: the agent finds them in a short time

\section{Aim}
\label{sec:aim}

% What is the underlying purpose of the thesis project?

This work tries to address these issues, focusing on strategic scans of larger environments where the field of view is small relative to the environment.
This is a problem that has been less studied in the literature than visual search in smaller environments.
There are other factors that become increasingly important. The field-of-view of the observer is often limited, and she has to move it efficiently to find the target.

% what will be done
The aim of this thesis is to implement and evaluate an autonomous agent that intelligently searches its environment for targets.
The agent should learn common characteristics of environments and utilize this knowledge to search for targets in new environments more effectively.
Furthermore, the agent should be able to generalize to unseen environments drawn from the same distribution as the ones it has seen previously.

% how it will be done
A specific instance of the visual search problem is considered, where the environment is searched by a pan-tilt camera fixed in place.
The camera has a limited view of the environent.
Automating this task is of interest for multiple reasons.
Manually controlling a camera may be costly, and the performance of a human operator may be suboptimal.
Crucial to the problem is generalization.

\section{Research questions}
\label{sec:research-questions}

% This is where the research questions are described.
% Formulate these as explicit questions, terminated with a
% question mark. A report will usually contain several different
% research questions that are somehow thematically connected.
% There are usually 2-4 questions in total.
% 
% Examples of common types of research questions (simplified
% and generalized):
% 
% \begin{enumerate}
% \item How does technique X affect the possibility of achieving the
%   effect Y?
% 
% \item How can a system (or a solution) for X be realized so
%   that the effect Y is achieved?
% 
% \item What are the alternatives to
%   achieving X, and which alternative gives the best effect considering
%   Y and Z? (This research question is normally broken down in to 2
%   separate questions.)
% 
% \end{enumerate}
% 
% 
% Observe that a very specific research question almost always
% leads to a better thesis report than a general research question
% (it is simply much more difficult to make something good
% from a general research question.)
% 
% The best way to achieve a really good and specific research
% question is to conduct a thorough literature review and get
% familiarized with related research and practice. This leads to
% ideas and terminology which allows one to express oneself
% with precision and also have something valuable to say in the
% discussion chapter. And once a detailed research question
% has been specified, it is much easier to establish a suitable
% method and thus carry out the actual thesis work much faster
% than when starting with a fairly general research question. In
% the end, it usually pays off to spend some extra time in the
% beginning working on the literature review. The thesis
% supervisor can be of assistance in deciding when the research
% question is sufficiently specific and well-grounded in related
% research.

This thesis will address the following questions:

\begin{enumerate}
  \item How can a learning agent that does efficient visual search in familiar environments be implemented?
  \item How well does the learning agent generalize to unseen but familiar environments?
  \item How does the learning agent compare to an exhaustive search of the environment, frontier-based algorithm, and a human searcher, and \textbf{other RL algorithm}?
  %\item How does the learning agent compare to common non-learning methods?
  %\item How can a simulator that tests the ability of an agent to solve the presented problem be implemented?
  %\item How does memory affect the agent's ability to search an environment?
\end{enumerate}

\section{Delimitations}
\label{sec:delimitations}

% This is where the main delimitations are described. For
% example, this could be that one has focused the study on a
% specific application domain or target user group. In the
% normal case, the delimitations need not be justified.

This thesis will be focused on the behavioral aspects of the presented problem.
We do not focus on difficult detection problems, but rather efficient actions.
To train and test agents, a simplified environment will be used. 
This will test the desired characteristics of the agent as presented above, but will not simulate realistic environments.
For simplicity, we assume that the environment is static. % explain
We also focus on the search process and not the detection, and therefore targets will be easy to detect once visible.

% Having a variable number of targets becomes problematic: an exhaustive search of the environment is necessary for the agent to know for certain if it is done.
