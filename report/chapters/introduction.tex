% should decide on a clear terminology,
% target, distractor, region, view, move, etc.

\chapter{Introduction}
\label{cha:introduction}

% ~3p

In this thesis project, the problem of searching for targets in unknown but familiar environments is addressed.
This chapter presents the motivation behind the project, the research questions that are addressed, and the delimitations. 

\section{Motivation}
\label{sec:motivation}

% This is where the studied problem is described from a general
% point of view and put in a context which makes it clear that
% it is interesting and well worth studying. The aim is to make
% the reader interested in the work and create an urge to
% continue reading.

The ability to visually search for things in an environment is fundamental to intelligent behavior.
We humans are constantly looking for things, be it be it the right book in the bookshelf, a certain keyword in an article or blueberries in the forest.
In many cases, it is important that this search is strategic, efficient, and fast.
Animals need to quickly identify predators, and drivers need to be able to search for pedestrians crossing the road they are driving on.

Automating visual search is of great interest for several reasons.
Visual search is crucial for applications such as search and rescue, surveillance, fire detection, etc.
Autonomous vehicles can both reduce risk and potentially exhibit more intelligent searching behavior than human-controlled ones.

However, while visual search is often seemingly effortless to us humans, it is a complex process.
Attempts to understand and recreate human visual search in machines has been a big challenge~\cite{eckstein_visual_2011}.
At the root of the visual search problem is partial observability.
A searcher can only perceive, or pay attention to, a limited region of the searched environment at once.
Therefore which regions to observe and in what order becomes an important decision. 

How humans and animals search for things has been extensively studied in neuroscience~\cite{eckstein_visual_2011,wolfe_visual_2010,nakayama_situating_2011}.
When we search, we use features of the environment to guide our attention~\cite{wolfe_five_2017,eckstein_visual_2011}.
For example, we know to look for berries at the forest floor, and not to look for boats on land.
Furthermore, search is not purely reactive but involves the use of memory.
We also use memory to take the history of the search into account when deciding where to move our attention~\cite{wolfe_five_2017}.

Such features can in some cases be quite subtle and difficult to pick up, even for humans.
Manually engineering guidance in accordance with these features can be difficult,
especially if a searching system should be deployed in many different environments.
Doing so manually can be labour intensive, especially if a searching system should be deployed in many different environments.
If one could instead learn the a good searching strategy from a limited set of sample environments this would be circumvented.
Such a system could be taught to search in arbitrary environments without the use of environment-specific rules.

Reinforcement learning~\cite{sutton_reinforcement_2018} is a paradigm that is suited for learning mappings from sensor values to actions.
In recent years, reinforcement learning has been combined with deep learning~\cite{goodfellow_deep_2016} with tremendous success.
It has been used to master arcade games~\cite{mnih_human_2015}, board games~\cite{silver_alphago_2016}, and even complex real-time strategy games~\cite{vinyals_alphastar_2019}.
Several works have also applied reinforcement learning to tasks involving embodied agents with visual input~\cite{minut_mahadevan_2001,mnih_attention_2014,zhu_target_driven_2016,mirowski_navigate_2017}.
This makes it interesting to see if deep reinforcement learning can also be applied to visual search.

\section{Aim}
\label{sec:aim}

The aim of this thesis is to investigate how an intelligent agent that learns to search for targets can be implemented with deep reinforcement learning.
Such an agent should learn the characteristics of the environments it is trained on and utilize this knowledge to effectively search for targets in unseen environments.
We postulate that an effective searcher 

\begin{itemize}
  \item prioritizes regions where the probability of finding a target is high according to previous experience,
  \item is able to search the environment exhaustively,
  \item avoids searching the same area twice unless,
  \item learns how the distribution of targets is correlated to the appearance of the environment,
  \item utilizes information from previously visited regions to decide where to look, and
  \item is able to find multiple targets while minimizing its path length. % test by comparing to optimal travel cost
\end{itemize}

We ask ourselves how an intelligent agent can be trained with reinforcement learning to search with these criteria in mind.
Specifically, we consider scenarios where the agent can only observe a small portion of its environment at any given time through a fixed pan-tilt camera.
The agent has to actively choose where to look in order to gain new information about the environment.
The goal of the agent is to actively locate a set of targets in its environment.

While similar problems have been addressed in the past~\cite{minut_mahadevan_2001}, our impression is that this is the first work to address intelligent search for multiple targets in unseen environments.
Our contributions are as follows:

\begin{itemize}
  %\item We formally introduce the visual search problem for reinforcement learning.
  %\item We discuss the problem from several angles.
  \item We provide a set of environments to evaluate visual search agents.
  \item We propose a method for solving the visual search task with reinforcement learning.
  \item We compare the method to a set of common baseline agents.
  \item We evaluate how well learning agents are able to generalize to unseen environments.
\end{itemize}

\section{Research questions}
\label{sec:research-questions}

% This is where the research questions are described.
% Formulate these as explicit questions, terminated with a
% question mark. A report will usually contain several different
% research questions that are somehow thematically connected.
% There are usually 2-4 questions in total.
% 
% Examples of common types of research questions (simplified
% and generalized):
% 
% \begin{enumerate}
% \item How does technique X affect the possibility of achieving the
%   effect Y?
% 
% \item How can a system (or a solution) for X be realized so
%   that the effect Y is achieved?
% 
% \item What are the alternatives to
%   achieving X, and which alternative gives the best effect considering
%   Y and Z? (This research question is normally broken down in to 2
%   separate questions.)
% 
% \end{enumerate}
% 
% 
% Observe that a very specific research question almost always
% leads to a better thesis report than a general research question
% (it is simply much more difficult to make something good
% from a general research question.)
% 
% The best way to achieve a really good and specific research
% question is to conduct a thorough literature review and get
% familiarized with related research and practice. This leads to
% ideas and terminology which allows one to express oneself
% with precision and also have something valuable to say in the
% discussion chapter. And once a detailed research question
% has been specified, it is much easier to establish a suitable
% method and thus carry out the actual thesis work much faster
% than when starting with a fairly general research question. In
% the end, it usually pays off to spend some extra time in the
% beginning working on the literature review. The thesis
% supervisor can be of assistance in deciding when the research
% question is sufficiently specific and well-grounded in related
% research.

This thesis will address the following questions:

\begin{enumerate}
  \item \label{itm:rq1} How can a learning agent that learns to intelligently search for targets be implemented?
  \item What is a suitable memory architecture for a visual search agent?
  \item \label{itm:rq2} How does the learning agent compare to random walk, exhaustive search, a human searcher?
  \item \label{itm:rq3} How many training samples are needed for the agent to generalize to unseen in-distribution environments?
\end{enumerate}

\section{Delimitations}
\label{sec:delimitations}

% This is where the main delimitations are described. For
% example, this could be that one has focused the study on a
% specific application domain or target user group. In the
% normal case, the delimitations need not be justified.

This thesis will be focused on the behavioral aspects of the presented problem.
We do not focus on difficult detection problems, but rather strategic actions.
For this reason, targets will deliberately be made easy to detect.
For simplicity, we make the assumption that the searched environment is static.
The appearance of the environment and the location of the targets does not change from one observation to the next.
