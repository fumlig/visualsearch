\chapter{Introduction}
\label{cha:introduction}

% unknown but familiar

In this thesis project, the problem of searching for targets in unknown but familiar environments is addressed. This chapter presents the motivation behind the project, the research questions that are addressed, and the delimitations. 

\section{Motivation}
\label{sec:motivation}

% general problem
Efficiently search for targets (objects of interest, goals) is an essential skill for interacting with the world. Humans and animals look for things.

% https://link.springer.com/content/pdf/10.3758/s13423-020-01859-9.pdf has some good examples

% time critical in some scenarious (conflict, driving a car, detecting predator)

% neuroscience
The problem has been studied continually under the name \textit{visual search} in neuroscience and neurobiology since YYYY. 
% no automation

% computer vision
In computer vision, a similar problem is studied under the name \textit{object localization}
% randomly sampled, no active aspect

% details our setup
In many real-world scenarios, the characteristics of the searched environment can be used to find targets quicker. For example, if one is to find an apple in a kitchen it makes sense to search the countertops for a fruit bowl. Similarly, ....

The exact characteristics of the environment need not be constant - for every kitchen is some prior knowledge we can use to find targets in it quicker. In many cases, the environment is familiar in that it has characteristics that are similar to previously seen environments.

% should decide ratio between field of view and scene
% small ratio: corresponds to looking at larger sceneries with a zoomed in camera (probably this)
% high ratio: more similar to human search, a lot is seen at once but expensive to process, requires attention, foveated vision etc.

% why automate
It is of great interest to learn how to automate visual search. Applications range from search and rescue to ...



\section{Aim}
\label{sec:aim}

% what will be done
The aim of this thesis is to implement and evaluate an autonomous agent that intelligently searches its environment for targets. The agent should learn common characteristics of environments and utilize this knowledge to search for targets in new environments more effectively.

% how it will be done
A specific instance of the visual search problem is considered, where the environment is searched by a pan-tilt camera fixed in place. The camera has a limited view of the environent. Automating this task is of interest for multiple reasons. Manually controlling a camera may be costly, and the performance of a human operator may be suboptimal. Crucial to the problem is generalization.

\section{Research questions}
\label{sec:research-questions}

This thesis will address the following questions:

\begin{enumerate}
  \item How can a learning agent that does efficient visual search in familiar environments be implemented?
  \item How can a simulator that tests the ability of an agent to solve the presented problem be implemented?
  \item How can a learning agent generalize to unseen but familiar environments?
  \item How does the learning agent compare to common non-learning methods?
\end{enumerate}

% think about measurements
% maybe don't have one for environment
% is a non-learning method feasible (which algorithms to use? depends on environment)
% i.e. what happens if target appearance is drawn from a distribution? can a rule based agent solve the problem?

\section{Delimitations}
\label{sec:delimitations}

This thesis will be focused on the behavioural aspects of the presented problem. To train and test agents, a simplified environment will be used. This will test the desired characteristics of the agent as presented above, but will not simulate realistic environments.

Object detection is not the primary focus of the project, ie. problems related to variability in target appearance are not considered (lightning, differing viewpoints, presence of obscuring objects, sensor noise, etc.).

% Do we assume that we know when we are looking at a target?
% Oddity search...

% Having a variable number of targets becomes problematic: an exhaustive search of the environment is necessary for the agent to know for certain if it is done.

% Maybe we assume that we have semantic meaning sorted out
% Can be motivated by referring to computer vision literature.