\chapter{Method}
\label{cha:method}

In this chapter, the method is described.

\section{Environment}

The environments to be searched are drawn from a distribution, with varying but similar appearance, target locations and appearances. For all environments, the appearance correlates to the probability of targets.

\subsection{Action Space}

\subsection{Observation Space}

\subsection{Reward Signal}

\subsection{Algorithm}

\subsection{Feature extraction}

\section{Experiments}

\subsection{Hyperparameters}

\subsection{Generalization}

\subsection{Memory}

% learning actions, recognition and localization simultaneously feels in itself interesting
% how do we interpret such a model? are there papers for visualizing reinforcement learning weights

% detailed
% replicability

% pre-study
% implementation
% evaluation

% discuss state representation
% need more than the image
% - either memory or position or both

% seed + 1000 levels training
% >= s + ... for test
% do the same as procgen

% also same configuration possibly?
% more parallel environments seem to lead to stable training (relatively slow increase in reward though...)

% would be nice to have a varying number of targets
% also a done action
% measure false positives, false negatives etc.
% good replacement for zoom?
% relates to visual search literature
% the current setup is also the most natural for a computer vision system
% foveated vision is not very reasonable


% select one algorithm and clearly motivate why!
% probably PPO, but why?


% time to find targets could be measured in terms of execution time, 
% and number of timesteps. this way, it could be easier to incorporate 
% visual attention if it is deemed reasonable.

% how do we handle appearance of targets:
% can an agent learn to recognize anything out of the ordinary?
% maybe not the focus, but just the search

% an interesting question is how large search windows we can handle
% could be an alternative to zoom
% a large search window presumably requires attention


% memory is needed for (a) remembering previous locations, and (b) integrating features over time


% PPO
% PPG