\chapter{Method}
\label{cha:method}

% detailed
% replicability

% pre-study
% implementation
% evaluation

% discuss state representation
% need more than the image
% - either memory or position or both

% seed + 1000 levels training
% >= s + ... for test
% do the same as procgen

% also same configuration possibly?
% more parallel environments seem to lead to stable training (relatively slow increase in reward though...)

% would be nice to have a varying number of targets
% also a done action
% measure false positives, false negatives etc.
% good replacement for zoom?
% relates to visual search literature
% the current setup is also the most natural for a computer vision system
% foveated vision is not very reasonable

In this chapter, the method is described in a way which shows how the
work was actually carried out. The description must be precise and
well thought through. Consider the scientific term
replicability. Replicability means that someone reading a scientific
report should be able to follow the method description and then carry
out the same study and check whether the results obtained are
similar. Achieving replicability is not always relevant, but precision
and clarity is.

Sometimes the work is separated into different parts, e.g.  pre-study,
implementation and evaluation. In such cases it is recommended that
the method chapter is structured accordingly with suitable named
sub-headings.
