\chapter{Results}
\label{cha:results}

% ~10p

% This chapter presents the results. Note that the results are presented
% factually, striving for objectivity as far as possible.  The results
% shall not be analyzed, discussed or evaluated.  This is left for the
% discussion chapter.
% 
% In case the method chapter has been divided into subheadings such as
% pre-study, implementation and evaluation, the result chapter should
% have the same sub-headings. This gives a clear structure and makes the
% chapter easier to write.
% 
% In case results are presented from a process (e.g. an implementation
% process), the main decisions made during the process must be clearly
% presented and justified. Normally, alternative attempts, etc, have
% already been described in the theory chapter, making it possible to
% refer to it as part of the justification.

%When it comes to hyperparameters, we find that letting the number of rollout steps be substantially lower than the episode length we achieve much more stable training results.
%Furthermore, increasing the number of weights in the neural network made it more difficult to train.

%We find that the hyperparameters from [procgen] perform well, especially when the number of environments is large.

%Also, proximal policy optimization was unstable without reward normalization. % Discussion: We hypothesize that this is because...

%This coupled with a sparse reward signal led to many cases where the agent converged towards a poor local optimum (or perhaps never converged at all).


This chapter presents the results for each of the experiments described in Section ~\ref{sec:experiments}.

\section{Scalability to Search Space Sizes}

Figure 

% 3 plots, one for each shape
% 2 algorithms per plot
% 5 seeds per plot with mean and standard deviation
% Check if test performance maxes out before

\begin{figure}
    \centering
    \label{fig:train-shape}
    \caption[Search space size learning curve.]{Reward and episode length curves during training for three different search space sizes. Mean and standard deviation across 3 seeds.}
\end{figure}

\begin{table}
    \caption[Search space performance metrics.]{Number of completed searches, average length of completed searches and SPL score.}
    \centering
    \label{tab:test-shape}
\end{table}

\section{Generalization with Limited Number of Samples}

\begin{figure}
    \centering
    \label{fig:samples}
    \caption[Generalization results.]{Reward and episode length curves during training for three different search space sizes. Mean and standard deviation across 3 seeds.}
\end{figure}


\section{Comparison to Human Search Performance}

\begin{itemize}
    \item 100 trials with N = 5 humans, N = 5 seeds.
    \item Add trees and hide objects behind them.
    \item Search space is no longer uniform, can afford larger spaces.
    \item Should we add zoom? Do we have time?
    \item Success rate, SPL, average length.
\end{itemize}