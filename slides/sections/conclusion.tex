\section{Conclusion}

\begin{frame}
    \frametitle{Conclusion}

    \begin{itemize}
        \item Architecture:
        \begin{itemize}
            \item Spatial memory: architecture scales to larger search spaces and generalizes better.
            \item Temporal memory: sufficient (and better) for smaller search spaces.
        \end{itemize}
        \item Approach:
        \begin{itemize}
            \item Search performance: better than simple baselines, comparable to human, worse than handcrafted.
            % was able to utilize visual cues to guide search.
            % Family of RL algorithms I have used known to converge to local optima.
            % Could a better reinforcement learning algorithm have been used?
            % There where some revisits: is stochastic policy suited?
            % Better than simple baselines, comparable to human, worse than handcrafted.
            % Does it fulfull the conditions that are needed to learn better behavior, or is something missing?
            \item Sample efficiency: relatively many samples needed even for simple environments.
            % sample efficiency: quite many samples where needed even for simple scenarios. known problem with deep RL.
        \end{itemize}
    \end{itemize}
\end{frame}

\subsection{Future Work}

\begin{frame}
    \frametitle{Future Work}

    \begin{itemize}
        \item Improvements to approach.
        \begin{itemize}
            \item Neural network architecture.
            \item Reinforcement learning algorithm.
            \item Reward signal design.
            % Less bias?
        \end{itemize}
        \item Evaluate on realistic search scenarios.
        % We have looked at simulated environments which are easier to control.
        % Easier to vary parameters to get a sense of what is feasible.
        % The final goal is to use this for realistic scenarios.
        % There will be higher variance, difficult patterns, detection problems, noise.
        % Should see if this approach scales to such scenarios.
        % Is it enough to simply increase number of parameters?
        % Does the approach has to be modified?
        % Example: use a pretrained object detection network?
        \item Formal verification (for security-critical applications).
        % Handcrafted systems can offer more guarantees.
        % It is important to verify autonomous systems before deploying them.
        % This is especially true for safety-critical applications.
        % Value alignment etc.
    \end{itemize}
\end{frame}