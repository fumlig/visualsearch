\section{Experiments}

\subsection{Search Performance}

\begin{frame}
    \frametitle{Experiment I: Search Performance}

    \begin{itemize}
        \item Compare to simple reference behaviors (baselines).
        \item Test on held out environment samples.
        \item Metrics:
        \begin{enumerate}
            \item Average search path length.
            \item Average success rate.
            \item Success weighted by inverse path length (SPL)~\cite{anderson_evaluation_2018}.
        \end{enumerate}
    \end{itemize}

    \begin{definition}
        \begin{equation*}
            \text{SPL} = \frac{1}{N} \sum_{i=1}^N S_i \frac{l_i}{\max(p_i,l_i)}
        \end{equation*}
        where \(N\) is number of test samples, \(S_i\) is binary success indicator, \(p_i\) is the taken path length \(l_i\) is the shortest path length.
    \end{definition}

    % How good was the taken search path compared to the optimal search path?
    % How good a path is depends on the environment --
    % if targets are close together, a good path should be short.
    % SPL is 1.0 of the agent takes the optimal path in all test samples
    % SPL is 0.5 if the agent takes the optimal path but is only successful in half of the tests.
    % etc.
\end{frame}

\begin{frame}
    \frametitle{Baselines}

    % Simple handcrafted policies.
    % Give a sense of the performance achieved by learning agents.
    % All indicate automatically when target visible.

    \begin{description}
        \item [Random:] randomly samples actions.
        \item [Greedy:] greedily selects exploring actions (random if none).
        \item [Exhaustive:] exhaustively covers search space with minimal revisits.
        \item [Human:] human searcher with knowledge of environment.
        \item[Handcrafted:] prioritize actions that lead to higher blue intensity\\ (gaussian environment only).
    \end{description}
\end{frame}

%\movie[externalviewer]{\includegraphics{../videos/gaussian/map/0.gif}}{videos/gaussian/map/0.gif}

\begin{frame}
    \begin{table}
        \centering
        Gaussian Environment\par\vspace{0.5em}
        \begin{tabular}{lccc}
    \toprule
    Agent & SPL & Length & Success \\
    \midrule
    random & $0.07 \pm 0.01$ & $417.69 \pm 30.87$ & $0.93 \pm 0.02$\\
    greedy & $0.15 \pm 0.00$ & $148.69 \pm 3.41$ & $1.00 \pm 0.01$\\
    exhaustive & $0.21 \pm 0.00$ & $82.10 \pm 3.64$ & $1.00 \pm 0.00$\\
    human & $0.23 \pm 0.03$ & $80.97 \pm 13.49$ & $1.00 \pm 0.00$\\
    lstm & $0.25 \pm 0.02$ & $108.26 \pm 12.67$ & $0.99 \pm 0.01$\\
    map & $0.30 \pm 0.02$ & $74.25 \pm 12.55$ & $1.00 \pm 0.01$\\
    \bottomrule
\end{tabular}

    \end{table} 

    \begin{center}
        \href{run:./videos/gaussian/map/0.mp4}{\texttt{video 1}},
        \href{run:./videos/gaussian/map/1.mp4}{\texttt{video 2}},
        \href{run:./videos/gaussian/map/2.mp4}{\texttt{video 3}}.
    \end{center}
\end{frame}

\begin{frame}
    \begin{table}
        \centering
        Terrain Environment\par\vspace{0.5em}
        \begin{tabular}{lccc}
    \toprule
    Agent & SPL & Length & Success \\
    \midrule
    random & $0.05 \pm 0.00$ & $360.08 \pm 19.08$ & $0.89 \pm 0.01$\\
    greedy & $0.14 \pm 0.00$ & $142.78 \pm 8.60$ & $1.00 \pm 0.00$\\
    exhaustive & $0.19 \pm 0.00$ & $81.91 \pm 1.13$ & $1.00 \pm 0.00$\\
    human & $0.26 \pm 0.02$ & $76.73 \pm 5.33$ & $1.00 \pm 0.00$\\
    \bottomrule
\end{tabular}
    \end{table}

    \begin{center}
        \href{run:./videos/terrain/map/0.mp4}{\texttt{video 1}},
        \href{run:./videos/terrain/map/1.mp4}{\texttt{video 2}},
        \href{run:./videos/terrain/map/2.mp4}{\texttt{video 3}}.
    \end{center}
\end{frame}

\begin{frame}
    \begin{table}
        \centering
        Camera Environment\par\vspace{0.5em}
        \begin{tabular}{lccc}
    \toprule
    Agent & SPL & Length & Success \\
    \midrule
    random & $0.03 \pm 0.01$ & $514.95 \pm 27.59$ & $0.63 \pm 0.08$\\
    greedy & $0.11 \pm 0.01$ & $253.42 \pm 25.14$ & $0.99 \pm 0.00$\\
    exhaustive & $0.34 \pm 0.00$ & $64.16 \pm 0.00$ & $1.00 \pm 0.00$\\
    human & $0.66 \pm 0.07$ & $35.50 \pm 5.45$ & $1.00 \pm 0.00$\\
    \bottomrule
\end{tabular}
    \end{table}

    \begin{center}
        \href{run:./videos/camera/lstm/1.mp4}{\texttt{video 1}},
        \href{run:./videos/camera/lstm/500.mp4}{\texttt{video 2}},
        \href{run:./videos/camera/lstm/1000.mp4}{\texttt{video 3}}.
    \end{center}
\end{frame}

\subsection{Scaling to Larger Search Spaces}

\begin{frame}
    \frametitle{Experiment II: Scaling to Larger Search Spaces}

    \begin{itemize}
        \item Larger search spaces are more difficult.
        \item Stronger demands on memory:
        \begin{itemize}
            \item Remember visited positions.
            \item Remember appearance of environment.
        \end{itemize}
        \item Compare memories on \(10 \times 10\), \(15 \times 15\), and \(20 \times 20\) versions of gaussian environment.
    \end{itemize}
\end{frame}

\begin{frame}
    \begin{figure}
        \centering
        \(10 \times 10\)
        \includegraphics[scale=0.8]{figures/shape-10.pdf}
    \end{figure}
\end{frame}

\begin{frame}
    \begin{figure}
        \centering
        \(15 \times 15\)
        \includegraphics[scale=0.8]{figures/shape-15.pdf}
    \end{figure}
\end{frame}

\begin{frame}
    \begin{figure}
        \centering
        \(20 \times 20\)
        \includegraphics[scale=0.8]{figures/shape-20.pdf}
    \end{figure}
\end{frame}

\subsection{Generalization From Limited Samples}

\begin{frame}
    \frametitle{Experiment III: Generalization From Limited Samples}

    \begin{itemize}
        \item Real-world tasks usually have limited training samples.
        \item Train on 500, 1 000, 5 000 and 10 000 samples of terrain environment.
        \item Test on held out samples from full distribution.
        % high appearance variance and somewhat realistic.
        %Fix seed pool used to generate scenes seen during training.
    \end{itemize}
\end{frame}

\begin{frame}
    \begin{figure}
        \centering
        10000 samples
        \includegraphics[scale=0.8]{figures/sample-10000.pdf}
    \end{figure}
    % close to full distribution - equal training and test performance for both agents. seemingly no overfitting and good generalization.
\end{frame}

\begin{frame}
    \begin{figure}
        \centering
        5000 samples
        \includegraphics[scale=0.8]{figures/sample-5000.pdf}
    \end{figure}
\end{frame}

\begin{frame}
    \begin{figure}
        \centering
        1000 samples
        \includegraphics[scale=0.8]{figures/sample-1000.pdf}
    \end{figure}
\end{frame}

\begin{frame}
    \begin{figure}
        \centering
        500 samples
        \includegraphics[scale=0.8]{figures/sample-500.pdf}
    \end{figure}
\end{frame}