
\section{Introduction}

\begin{frame}
    \frametitle{Description}
    
    Learned autonomous search for a set of targets in a visual environment with a camera.

    \begin{itemize}
        \item Camera views limited region of environment.
        \item Moving camera changes visible region.
        \item Detect when targets are visible.
        \item Locate targets in minimum time.
        \item Utilize visual cues to find targets quicker.
        \item Learn control from a set of sample scenarios.
        \item Use deep reinforcement learning.
    \end{itemize}
\end{frame}

\begin{frame}
    \frametitle{Aspects}

    \begin{itemize}
        \item Random or exhaustive search sufficient in small or random environments~\cite{nakayama_situating_2011}.
        \item Most real-world search tasks exhibit structure.
        \item Visual cues can be used to find targets quicker.
        \begin{itemize}
            \item Books are in bookshelves, cars on roads\dots
            \item Targets spread out/close together\dots
        \end{itemize}
        \item Patterns and cues may be subtle and difficult to pick up.
    \end{itemize}
\end{frame}

\begin{frame}
    \frametitle{Motivation}
    
    \begin{itemize}
        \item Autonomous systems may reduce risk and cost.
        \item Applications in search and rescue, surveillance, home assistance, etc.
        \item Handcrafted systems using domain knowledge are difficult to design.
        \item Subtle patterns and difficult planning decisions.
        \item Learning system applicable as long as data is available.
    \end{itemize}
\end{frame}

\begin{frame}
    \frametitle{Design Challenges}

    \begin{itemize}
        \item Prioritize regions with high probability of targets based on previous experience.
        \item Learn correlations between scene appearance and target probability.
        \item Search exhaustively while avoiding searching the same region twice.
        \item Remember features of searched regions (avoid revisits, scene understanding).
        \item Real-world tasks have limited number of training samples.
    \end{itemize}
\end{frame}

\begin{frame}
    \frametitle{Research Questions}
    \begin{enumerate}
        \item How can an agent that learns to intelligently search for targets be implemented with deep reinforcement learning?
        \item How does the learning agent compare to random walk, exhaustive search, and a human searcher with prior knowledge of the searched scenes?
        \item How well does the learning agent generalize from a limited number of training samples to unseen in-distribution search scenarios?
    \end{enumerate}    
\end{frame}